\subsection*{Programme qui synchronise un dossier local de notre machine vers un site miroir ftp distant}

Python course -\/ T\+P03

rendu le 18 au soir au plus tard pdf + zip

\subsection*{parametres}

\tabulinesep=1mm
\begin{longtabu} spread 0pt [c]{*3{|X[-1]}|}
\hline
\rowcolor{\tableheadbgcolor}{\bf Paramètre}&{\bf Type}&{\bf Variable  }\\\cline{1-3}
\endfirsthead
\hline
\endfoot
\hline
\rowcolor{\tableheadbgcolor}{\bf Paramètre}&{\bf Type}&{\bf Variable  }\\\cline{1-3}
\endhead
site ftp distant&obligatoire&ftp \\\cline{1-3}
chemin vers le dossier local (directory path)&obligatoire&dp \\\cline{1-3}
chemin pour generer le fichier log (log path)&obligatoire&lp \\\cline{1-3}
2-\/uple contenant les extensions de la liste de fichiers a inclure et de la liste de fichiers a exclure&obligatoire&ie \\\cline{1-3}
chemin vers le fichier conf du log (gestion des handler)&optionnel&\char`\"{}-\/lc\char`\"{}, \char`\"{}-\/-\/log\+Conf\char`\"{} \\\cline{1-3}
profondeur de la supervision du dossier, default = 2&optionnel&\char`\"{}-\/p\char`\"{}, \char`\"{}-\/-\/profondeur\char`\"{} \\\cline{1-3}
taille maximale des fichiers transferes en Mo, default = 500 Mo&optionnel&\char`\"{}-\/sf\char`\"{},\char`\"{}-\/-\/size\+File\char`\"{} \\\cline{1-3}
frequence de supervision en s, default = 1 s&optionnel&\char`\"{}-\/f\char`\"{}, \char`\"{}-\/-\/frequence\char`\"{} \\\cline{1-3}
temps de supervision en s, default = 60 sec&optionnel&\char`\"{}-\/st\char`\"{}, \char`\"{}-\/-\/supervision\+Time\char`\"{} \\\cline{1-3}
\end{longtabu}


commande linux = rsync

paramètre en lignes de commande ou dans un fichier ini

\subsection*{Note explicative}

Un pdf qui permet d\textquotesingle{}expliquer notre projet, pourquoi on a choisi de faire le truc comme ça. Faut vendre le projet.


\begin{DoxyItemize}
\item auteurs
\item Si librairie supplémentaire =$>$ installable par pip
\item choix techniques
\item choix d\textquotesingle{}organisation
\end{DoxyItemize}

En lignes de commande 